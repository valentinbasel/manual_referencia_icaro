% Generated by Sphinx.
\def\sphinxdocclass{report}
\newif\ifsphinxKeepOldNames \sphinxKeepOldNamestrue
\documentclass[letterpaper,10pt,english]{sphinxmanual}
\usepackage{iftex}

\ifPDFTeX
  \usepackage[utf8]{inputenc}
\fi
\ifdefined\DeclareUnicodeCharacter
  \DeclareUnicodeCharacter{00A0}{\nobreakspace}
\fi
\usepackage{cmap}
\usepackage[T1]{fontenc}
\usepackage{amsmath,amssymb,amstext}
\usepackage{babel}
\usepackage{times}
\usepackage[Bjarne]{fncychap}
\usepackage{longtable}
\usepackage{sphinx}
\usepackage{multirow}
\usepackage{eqparbox}


\addto\captionsenglish{\renewcommand{\figurename}{Fig.\@ }}
\addto\captionsenglish{\renewcommand{\tablename}{Table }}
\SetupFloatingEnvironment{literal-block}{name=Listing }

\addto\extrasenglish{\def\pageautorefname{page}}

\setcounter{tocdepth}{1}


\title{Manual de ensamblado de la placa NP07}
\date{Oct 23, 2016}
\release{0}
\author{Mateo Carabajal}
\newcommand{\sphinxlogo}{}
\renewcommand{\releasename}{Release}
\makeindex

\makeatletter
\def\PYG@reset{\let\PYG@it=\relax \let\PYG@bf=\relax%
    \let\PYG@ul=\relax \let\PYG@tc=\relax%
    \let\PYG@bc=\relax \let\PYG@ff=\relax}
\def\PYG@tok#1{\csname PYG@tok@#1\endcsname}
\def\PYG@toks#1+{\ifx\relax#1\empty\else%
    \PYG@tok{#1}\expandafter\PYG@toks\fi}
\def\PYG@do#1{\PYG@bc{\PYG@tc{\PYG@ul{%
    \PYG@it{\PYG@bf{\PYG@ff{#1}}}}}}}
\def\PYG#1#2{\PYG@reset\PYG@toks#1+\relax+\PYG@do{#2}}

\expandafter\def\csname PYG@tok@gd\endcsname{\def\PYG@tc##1{\textcolor[rgb]{0.63,0.00,0.00}{##1}}}
\expandafter\def\csname PYG@tok@gu\endcsname{\let\PYG@bf=\textbf\def\PYG@tc##1{\textcolor[rgb]{0.50,0.00,0.50}{##1}}}
\expandafter\def\csname PYG@tok@gt\endcsname{\def\PYG@tc##1{\textcolor[rgb]{0.00,0.27,0.87}{##1}}}
\expandafter\def\csname PYG@tok@gs\endcsname{\let\PYG@bf=\textbf}
\expandafter\def\csname PYG@tok@gr\endcsname{\def\PYG@tc##1{\textcolor[rgb]{1.00,0.00,0.00}{##1}}}
\expandafter\def\csname PYG@tok@cm\endcsname{\let\PYG@it=\textit\def\PYG@tc##1{\textcolor[rgb]{0.25,0.50,0.56}{##1}}}
\expandafter\def\csname PYG@tok@vg\endcsname{\def\PYG@tc##1{\textcolor[rgb]{0.73,0.38,0.84}{##1}}}
\expandafter\def\csname PYG@tok@vi\endcsname{\def\PYG@tc##1{\textcolor[rgb]{0.73,0.38,0.84}{##1}}}
\expandafter\def\csname PYG@tok@mh\endcsname{\def\PYG@tc##1{\textcolor[rgb]{0.13,0.50,0.31}{##1}}}
\expandafter\def\csname PYG@tok@cs\endcsname{\def\PYG@tc##1{\textcolor[rgb]{0.25,0.50,0.56}{##1}}\def\PYG@bc##1{\setlength{\fboxsep}{0pt}\colorbox[rgb]{1.00,0.94,0.94}{\strut ##1}}}
\expandafter\def\csname PYG@tok@ge\endcsname{\let\PYG@it=\textit}
\expandafter\def\csname PYG@tok@vc\endcsname{\def\PYG@tc##1{\textcolor[rgb]{0.73,0.38,0.84}{##1}}}
\expandafter\def\csname PYG@tok@il\endcsname{\def\PYG@tc##1{\textcolor[rgb]{0.13,0.50,0.31}{##1}}}
\expandafter\def\csname PYG@tok@go\endcsname{\def\PYG@tc##1{\textcolor[rgb]{0.20,0.20,0.20}{##1}}}
\expandafter\def\csname PYG@tok@cp\endcsname{\def\PYG@tc##1{\textcolor[rgb]{0.00,0.44,0.13}{##1}}}
\expandafter\def\csname PYG@tok@gi\endcsname{\def\PYG@tc##1{\textcolor[rgb]{0.00,0.63,0.00}{##1}}}
\expandafter\def\csname PYG@tok@gh\endcsname{\let\PYG@bf=\textbf\def\PYG@tc##1{\textcolor[rgb]{0.00,0.00,0.50}{##1}}}
\expandafter\def\csname PYG@tok@ni\endcsname{\let\PYG@bf=\textbf\def\PYG@tc##1{\textcolor[rgb]{0.84,0.33,0.22}{##1}}}
\expandafter\def\csname PYG@tok@nl\endcsname{\let\PYG@bf=\textbf\def\PYG@tc##1{\textcolor[rgb]{0.00,0.13,0.44}{##1}}}
\expandafter\def\csname PYG@tok@nn\endcsname{\let\PYG@bf=\textbf\def\PYG@tc##1{\textcolor[rgb]{0.05,0.52,0.71}{##1}}}
\expandafter\def\csname PYG@tok@no\endcsname{\def\PYG@tc##1{\textcolor[rgb]{0.38,0.68,0.84}{##1}}}
\expandafter\def\csname PYG@tok@na\endcsname{\def\PYG@tc##1{\textcolor[rgb]{0.25,0.44,0.63}{##1}}}
\expandafter\def\csname PYG@tok@nb\endcsname{\def\PYG@tc##1{\textcolor[rgb]{0.00,0.44,0.13}{##1}}}
\expandafter\def\csname PYG@tok@nc\endcsname{\let\PYG@bf=\textbf\def\PYG@tc##1{\textcolor[rgb]{0.05,0.52,0.71}{##1}}}
\expandafter\def\csname PYG@tok@nd\endcsname{\let\PYG@bf=\textbf\def\PYG@tc##1{\textcolor[rgb]{0.33,0.33,0.33}{##1}}}
\expandafter\def\csname PYG@tok@ne\endcsname{\def\PYG@tc##1{\textcolor[rgb]{0.00,0.44,0.13}{##1}}}
\expandafter\def\csname PYG@tok@nf\endcsname{\def\PYG@tc##1{\textcolor[rgb]{0.02,0.16,0.49}{##1}}}
\expandafter\def\csname PYG@tok@si\endcsname{\let\PYG@it=\textit\def\PYG@tc##1{\textcolor[rgb]{0.44,0.63,0.82}{##1}}}
\expandafter\def\csname PYG@tok@s2\endcsname{\def\PYG@tc##1{\textcolor[rgb]{0.25,0.44,0.63}{##1}}}
\expandafter\def\csname PYG@tok@nt\endcsname{\let\PYG@bf=\textbf\def\PYG@tc##1{\textcolor[rgb]{0.02,0.16,0.45}{##1}}}
\expandafter\def\csname PYG@tok@nv\endcsname{\def\PYG@tc##1{\textcolor[rgb]{0.73,0.38,0.84}{##1}}}
\expandafter\def\csname PYG@tok@s1\endcsname{\def\PYG@tc##1{\textcolor[rgb]{0.25,0.44,0.63}{##1}}}
\expandafter\def\csname PYG@tok@ch\endcsname{\let\PYG@it=\textit\def\PYG@tc##1{\textcolor[rgb]{0.25,0.50,0.56}{##1}}}
\expandafter\def\csname PYG@tok@m\endcsname{\def\PYG@tc##1{\textcolor[rgb]{0.13,0.50,0.31}{##1}}}
\expandafter\def\csname PYG@tok@gp\endcsname{\let\PYG@bf=\textbf\def\PYG@tc##1{\textcolor[rgb]{0.78,0.36,0.04}{##1}}}
\expandafter\def\csname PYG@tok@sh\endcsname{\def\PYG@tc##1{\textcolor[rgb]{0.25,0.44,0.63}{##1}}}
\expandafter\def\csname PYG@tok@ow\endcsname{\let\PYG@bf=\textbf\def\PYG@tc##1{\textcolor[rgb]{0.00,0.44,0.13}{##1}}}
\expandafter\def\csname PYG@tok@sx\endcsname{\def\PYG@tc##1{\textcolor[rgb]{0.78,0.36,0.04}{##1}}}
\expandafter\def\csname PYG@tok@bp\endcsname{\def\PYG@tc##1{\textcolor[rgb]{0.00,0.44,0.13}{##1}}}
\expandafter\def\csname PYG@tok@c1\endcsname{\let\PYG@it=\textit\def\PYG@tc##1{\textcolor[rgb]{0.25,0.50,0.56}{##1}}}
\expandafter\def\csname PYG@tok@o\endcsname{\def\PYG@tc##1{\textcolor[rgb]{0.40,0.40,0.40}{##1}}}
\expandafter\def\csname PYG@tok@kc\endcsname{\let\PYG@bf=\textbf\def\PYG@tc##1{\textcolor[rgb]{0.00,0.44,0.13}{##1}}}
\expandafter\def\csname PYG@tok@c\endcsname{\let\PYG@it=\textit\def\PYG@tc##1{\textcolor[rgb]{0.25,0.50,0.56}{##1}}}
\expandafter\def\csname PYG@tok@mf\endcsname{\def\PYG@tc##1{\textcolor[rgb]{0.13,0.50,0.31}{##1}}}
\expandafter\def\csname PYG@tok@err\endcsname{\def\PYG@bc##1{\setlength{\fboxsep}{0pt}\fcolorbox[rgb]{1.00,0.00,0.00}{1,1,1}{\strut ##1}}}
\expandafter\def\csname PYG@tok@mb\endcsname{\def\PYG@tc##1{\textcolor[rgb]{0.13,0.50,0.31}{##1}}}
\expandafter\def\csname PYG@tok@ss\endcsname{\def\PYG@tc##1{\textcolor[rgb]{0.32,0.47,0.09}{##1}}}
\expandafter\def\csname PYG@tok@sr\endcsname{\def\PYG@tc##1{\textcolor[rgb]{0.14,0.33,0.53}{##1}}}
\expandafter\def\csname PYG@tok@mo\endcsname{\def\PYG@tc##1{\textcolor[rgb]{0.13,0.50,0.31}{##1}}}
\expandafter\def\csname PYG@tok@kd\endcsname{\let\PYG@bf=\textbf\def\PYG@tc##1{\textcolor[rgb]{0.00,0.44,0.13}{##1}}}
\expandafter\def\csname PYG@tok@mi\endcsname{\def\PYG@tc##1{\textcolor[rgb]{0.13,0.50,0.31}{##1}}}
\expandafter\def\csname PYG@tok@kn\endcsname{\let\PYG@bf=\textbf\def\PYG@tc##1{\textcolor[rgb]{0.00,0.44,0.13}{##1}}}
\expandafter\def\csname PYG@tok@cpf\endcsname{\let\PYG@it=\textit\def\PYG@tc##1{\textcolor[rgb]{0.25,0.50,0.56}{##1}}}
\expandafter\def\csname PYG@tok@kr\endcsname{\let\PYG@bf=\textbf\def\PYG@tc##1{\textcolor[rgb]{0.00,0.44,0.13}{##1}}}
\expandafter\def\csname PYG@tok@s\endcsname{\def\PYG@tc##1{\textcolor[rgb]{0.25,0.44,0.63}{##1}}}
\expandafter\def\csname PYG@tok@kp\endcsname{\def\PYG@tc##1{\textcolor[rgb]{0.00,0.44,0.13}{##1}}}
\expandafter\def\csname PYG@tok@w\endcsname{\def\PYG@tc##1{\textcolor[rgb]{0.73,0.73,0.73}{##1}}}
\expandafter\def\csname PYG@tok@kt\endcsname{\def\PYG@tc##1{\textcolor[rgb]{0.56,0.13,0.00}{##1}}}
\expandafter\def\csname PYG@tok@sc\endcsname{\def\PYG@tc##1{\textcolor[rgb]{0.25,0.44,0.63}{##1}}}
\expandafter\def\csname PYG@tok@sb\endcsname{\def\PYG@tc##1{\textcolor[rgb]{0.25,0.44,0.63}{##1}}}
\expandafter\def\csname PYG@tok@k\endcsname{\let\PYG@bf=\textbf\def\PYG@tc##1{\textcolor[rgb]{0.00,0.44,0.13}{##1}}}
\expandafter\def\csname PYG@tok@se\endcsname{\let\PYG@bf=\textbf\def\PYG@tc##1{\textcolor[rgb]{0.25,0.44,0.63}{##1}}}
\expandafter\def\csname PYG@tok@sd\endcsname{\let\PYG@it=\textit\def\PYG@tc##1{\textcolor[rgb]{0.25,0.44,0.63}{##1}}}

\def\PYGZbs{\char`\\}
\def\PYGZus{\char`\_}
\def\PYGZob{\char`\{}
\def\PYGZcb{\char`\}}
\def\PYGZca{\char`\^}
\def\PYGZam{\char`\&}
\def\PYGZlt{\char`\<}
\def\PYGZgt{\char`\>}
\def\PYGZsh{\char`\#}
\def\PYGZpc{\char`\%}
\def\PYGZdl{\char`\$}
\def\PYGZhy{\char`\-}
\def\PYGZsq{\char`\'}
\def\PYGZdq{\char`\"}
\def\PYGZti{\char`\~}
% for compatibility with earlier versions
\def\PYGZat{@}
\def\PYGZlb{[}
\def\PYGZrb{]}
\makeatother

\renewcommand\PYGZsq{\textquotesingle}

\begin{document}

\maketitle
\tableofcontents
\phantomsection\label{index::doc}


Contents:


\chapter{placa robotica \emph{np07}}
\label{np07:placa-robotica-pcb}\label{np07:welcome-to-proyecto-icaro-documentacion-s-documentation}\label{np07::doc}
El hardware \emph{np07} esta basado en el micro controlador 18f4550 (o 18f2550), un
hardware con un bootloader y librerias usados en el proyecto PINGUINO.


\section{caracteristicas tecnicas}
\label{np07:caracteristicas-tecnicas}
la placa \emph{np07} cuentac con:
- 8 conversores analogicos (el micro controlador soporta 10)
- 4 entradas para sensores digitales
- 8 salidas digitales al PORTB (UNL2803)
- 2 salidas para motores de corriente continua ( L293B)
- 5 salidas de PWM para Servos (el micro controlador soporta hasa 18)


\section{listado de componentes}
\label{np07:listado-de-componentes}
\noindent\begin{tabulary}{\linewidth}{|L|L|L|L|}
\hline
\textsf{\relax 
Cantidad
\unskip}\relax &\textsf{\relax 
Componente
\unskip}\relax &\textsf{\relax 
Ubicación
\unskip}\relax &\textsf{\relax 
imagen
\unskip}\relax \\
\hline
11
&
Resistencias 470 Ohm - 1/4W
&
R1 R2 R3 R4 R5 R6 R7 R8 R9 R12 R17
&
\sphinxincludegraphics[width=40pt,height=40pt]{{resistencia-10k}.jpg}
\\
\hline
5
&
Resistencias 10k Ohm - 1/4W
&
R11 R13 R14 R15 R16
&
\sphinxincludegraphics[width=40pt,height=40pt]{{resistencia-470}.jpg}
\\
\hline
2
&
Capacitores Cerámicos 22pF
&
C2 C3
&
\sphinxincludegraphics[width=40pt,height=40pt]{{capacitor-22pf}.jpg}
\\
\hline
5
&
Capacitores Cerámicos 0.1uF
&
C9 C10 C11 (C12 C13)*
&
\sphinxincludegraphics[width=40pt,height=40pt]{{capacitor-01uf}.jpg}
\\
\hline
1
&
Capacitor Cerámico 220nF
&
C1
&
\sphinxincludegraphics[width=40pt,height=40pt]{{capacitor-220nf}.jpg}
\\
\hline
1
&
Capacitor Electrol. 10uF 16V
&
C5
&
\sphinxincludegraphics[width=40pt,height=40pt]{{capacitorelectrolitico}.jpg}
\\
\hline
4
&
Capacitor Electrol. 100uF
&
C4 C6 C7 C8
&
\sphinxincludegraphics[width=40pt,height=40pt]{{capacitor-100uf}.jpg}
\\
\hline
3
&
Diodos 1N4007
&
D9 D12 D14
&
\sphinxincludegraphics[width=40pt,height=40pt]{{diodo-1n4007}.jpg}
\\
\hline
11
&
Leds difusos 5mm
&
D1 D2 D3 D4 D5 D6 D7 D8 D10 D11 D12
&
\sphinxincludegraphics[width=40pt,height=40pt]{{led-difusos-5mm}.jpg}
\\
\hline
1
&
Conector USB hembra Tipo B
&
J1
&
\sphinxincludegraphics[width=40pt,height=40pt]{{conector-usb-b}.jpg}
\\
\hline\end{tabulary}


\noindent\begin{tabulary}{\linewidth}{|L|L|L|L|}
\hline
\textsf{\relax 
Cantidad
\unskip}\relax &\textsf{\relax 
Componente
\unskip}\relax &\textsf{\relax 
Ubicación
\unskip}\relax &\textsf{\relax 
imagen
\unskip}\relax \\
\hline
1
&
Push Button (Soft Touch)
&
SW2
&
\sphinxincludegraphics[width=40pt,height=40pt]{{pushbutton}.jpg}
\\
\hline
1
&
Regulador de Voltaje LM7805
&
U4
&
\sphinxincludegraphics[width=40pt,height=40pt]{{lm7805}.jpg}
\\
\hline
1
&
Regulador de Voltaje 78L05
&
U5
&
\sphinxincludegraphics[width=40pt,height=40pt]{{78L05}.jpg}
\\
\hline
7
&
Borneras Dobles
&
P8 P9 P10 P11 P12 P13 P14
&
\sphinxincludegraphics[width=40pt,height=40pt]{{bornera}.jpg}
\\
\hline
1
&
Zócalo de 8x2 Pines
&
U3
&
\sphinxincludegraphics[width=40pt,height=40pt]{{zocalo-8}.jpg}
\\
\hline
1
&
Zócalo de 20x2 Pines
&
U2
&
\sphinxincludegraphics[width=40pt,height=40pt]{{zocalo-20}.jpg}
\\
\hline
1
&
Zócalo de 9x2 Pines
&
P6
&
\sphinxincludegraphics[width=40pt,height=40pt]{{zocalo-9}.jpg}
\\
\hline
1
&
Cristal de 20Mhz
&
X1
&
\sphinxincludegraphics[width=40pt,height=40pt]{{cristal-20mhz}.jpg}
\\
\hline
2
&
Tira Postes Macho de 40 Pines
&
K2 K3 K4 K5 K6 SW1 SW3 K1 K8 P4
&
\sphinxincludegraphics[width=40pt,height=40pt]{{pinesmacho}.jpg}
\\
\hline
1
&
Tira de Postes Hembra de 40 Pines
&
P1 P7 P5 P15 P16 P17 P18
&
\sphinxincludegraphics[width=40pt,height=40pt]{{pineshembra}.jpg}
\\
\hline
1
&
Driver L293D (Puente H)
&
U3
&
\sphinxincludegraphics[width=40pt,height=40pt]{{L293D}.jpg}
\\
\hline
1
&
Integrado ULN2803
&
P6
&
\sphinxincludegraphics[width=40pt,height=40pt]{{uln2803}.jpg}
\\
\hline
1
&
Microcontrolador PIC18F4550
&
U2
&
\sphinxincludegraphics[width=40pt,height=40pt]{{pic18f4550}.jpg}
\\
\hline
4
&
jumper
&
SW1 SW3 K1 K8
&
\sphinxincludegraphics[width=40pt,height=40pt]{{Jumper}.jpg}
\\
\hline\end{tabulary}



\section{herramientas}
\label{np07:herramientas}
Las herramientas que necesitamos para armar una placa robotica \emph{np07}
son faciles de conseguir y muy comunes para cualquier
hobbista de la electronica.
\begin{figure}[htbp]
\centering
\capstart

\noindent\sphinxincludegraphics[width=300pt]{{soldador}.png}
\caption{Soldador}\label{np07:id1}\end{figure}

Un soldador eléctrico o de estaño, también conocido como cautín, es
una herramienta eléctrica usada para soldar. Funciona convirtiendo
la energía eléctrica en calor, que a su vez provoca
la fusión del material utilizado en la soldadura, como por
ejemplo el estaño.
\newpage\begin{figure}[htbp]
\centering
\capstart

\noindent\sphinxincludegraphics[width=300pt]{{estanio}.png}
\caption{Estaño}\label{np07:id2}\end{figure}

El estaño que se utiliza en electrónica tiene alma de resina con el fin
de facilitar la soldadura. Para garantizar una buena soldadura es
necesario que tanto el estaño como el elemento a soldar alcancen una
temperatura determinada, si esta temperatura no se alcanza se produce
el fenómeno denominado soldadura fría. La temperatura de fusión
depende de la aleación utilizada, cuyo componente principal es
el estaño y suele estar comprendida entre unos 200 a 400 ºC.

En realidad, el término ``estaño'' se emplea de forma impropia
porque no se trata de estaño sólo, sino de una aleación de este metal
con plomo, generalmente con una proporción respectiva
del 60\% y del 40\%, que resulta ser la más indicada para
las soldaduras en Electrónica.

Para realizar una buena soldadura, además del soldador
y de la aleación descrita, se necesita una sustancia adicional,
llamada pasta de soldar, cuya misión es la de facilitar la distribución
uniforme del estaño sobre las superficies a unir y evitando, al mismo
tiempo, la oxidación producida por la temperatura demasiado elevada
del soldador. La composición de esta pasta es a base de colofonia
(normalmente llamada ``resina'') y que en el caso del estaño que
utilizaremos, está contenida dentro de las cavidades del hilo,
en una proporción del 2\textasciitilde{}2.5\%.
\newpage\begin{figure}[htbp]
\centering
\capstart

\noindent\sphinxincludegraphics[width=300pt]{{alicate}.png}
\caption{alicate para electronica}\label{np07:id3}\end{figure}

Un pequeño alicate, para poder cortar el excedente de material (estaño,
alambres de las resistensias por ejmplo).
\newpage\begin{figure}[htbp]
\centering
\capstart

\noindent\sphinxincludegraphics[width=300pt]{{destornillador}.png}
\caption{destornillador plano pequeño}\label{np07:id4}\end{figure}

Nos sirve para ajustar las borneras y para hacer palanca para sacar un
integrado que hayamos puesto en un zocalo.
\newpage\begin{figure}[htbp]
\centering
\capstart

\noindent\sphinxincludegraphics[width=300pt]{{desoldador}.png}
\caption{desoldador de estaño}\label{np07:id5}\end{figure}

El desoldador de estaño, nos permite sacar el estaño que hayamos puesto
de mas o para remplazar algun componente efectuoso de la placa robotica \emph{np07}
\newpage

\section{fabricacion}
\label{np07:fabricacion}
A continución veremos el paso a paso del armado de la placa \emph{np07}.


\subsection{paso 0}
\label{np07:paso-0}\begin{figure}[htbp]
\centering
\capstart

\noindent\sphinxincludegraphics[width=300pt]{{0b}.jpg}
\caption{Vista de la Placa}\label{np07:id6}\end{figure}
\newpage

\subsection{paso 1}
\label{np07:paso-1}\begin{figure}[htbp]
\centering
\capstart

\noindent\sphinxincludegraphics[width=300pt]{{1b}.jpg}
\caption{Colocar 5 Puentes}\label{np07:id7}\end{figure}
\newpage

\subsection{paso 2}
\label{np07:paso-2}\begin{figure}[htbp]
\centering
\capstart

\noindent\sphinxincludegraphics[width=300pt]{{2b}.jpg}
\caption{Resistencias de 470 Ohm}\label{np07:id8}\end{figure}
\newpage

\subsection{paso 3}
\label{np07:paso-3}\begin{figure}[htbp]
\centering
\capstart

\noindent\sphinxincludegraphics[width=300pt]{{3b}.jpg}
\caption{Resistencias de 10K Ohm}\label{np07:id9}\end{figure}
\newpage

\subsection{paso 4}
\label{np07:paso-4}\begin{figure}[htbp]
\centering
\capstart

\noindent\sphinxincludegraphics[width=300pt]{{4b}.jpg}
\caption{Diodos 1N4007}\label{np07:id10}\end{figure}
\newpage

\subsection{paso 5}
\label{np07:paso-5}\begin{figure}[htbp]
\centering
\capstart

\noindent\sphinxincludegraphics[width=300pt]{{5b}.jpg}
\caption{Cristal de 20MHz}\label{np07:id11}\end{figure}
\newpage

\subsection{paso 6}
\label{np07:paso-6}\begin{figure}[htbp]
\centering
\capstart

\noindent\sphinxincludegraphics[width=300pt]{{6b}.jpg}
\caption{Capacitores Cerámicos 0,1uF}\label{np07:id12}\end{figure}
\newpage

\subsection{paso 7}
\label{np07:paso-7}\begin{figure}[htbp]
\centering
\capstart

\noindent\sphinxincludegraphics[width=300pt]{{7b}.jpg}
\caption{Capacitores Cerámicos 22pF}\label{np07:id13}\end{figure}
\newpage

\subsection{paso 8}
\label{np07:paso-8}\begin{figure}[htbp]
\centering
\capstart

\noindent\sphinxincludegraphics[width=300pt]{{8b}.jpg}
\caption{Capacitor Cerámico 220nF}\label{np07:id14}\end{figure}
\newpage

\subsection{paso 9}
\label{np07:paso-9}\begin{figure}[htbp]
\centering
\capstart

\noindent\sphinxincludegraphics[width=300pt]{{9b}.jpg}
\caption{Regulador LM7805}\label{np07:id15}\end{figure}
\newpage

\subsection{paso 10}
\label{np07:paso-10}\begin{figure}[htbp]
\centering
\capstart

\noindent\sphinxincludegraphics[width=300pt]{{10b}.jpg}
\caption{Regulador 78L05}\label{np07:id16}\end{figure}
\newpage

\subsection{paso 11}
\label{np07:paso-11}\begin{figure}[htbp]
\centering
\capstart

\noindent\sphinxincludegraphics[width=300pt]{{11b}.jpg}
\caption{Colocar Zócalos}\label{np07:id17}\end{figure}
\newpage

\subsection{paso 12}
\label{np07:paso-12}\begin{figure}[htbp]
\centering
\capstart

\noindent\sphinxincludegraphics[width=300pt]{{12b}.jpg}
\caption{Push Button}\label{np07:id18}\end{figure}
\newpage

\subsection{paso 13}
\label{np07:paso-13}\begin{figure}[htbp]
\centering
\capstart

\noindent\sphinxincludegraphics[width=300pt]{{13b}.jpg}
\caption{Colocar LEDS}\label{np07:id19}\end{figure}
\newpage

\subsection{paso 14}
\label{np07:paso-14}\begin{figure}[htbp]
\centering
\capstart

\noindent\sphinxincludegraphics[width=300pt]{{14b}.jpg}
\caption{Capacitores Electrolíticos 100uF}\label{np07:id20}\end{figure}
\newpage

\subsection{paso 15}
\label{np07:paso-15}\begin{figure}[htbp]
\centering
\capstart

\noindent\sphinxincludegraphics[width=300pt]{{15b}.jpg}
\caption{Capacitor Electrolítico 10uF}\label{np07:id21}\end{figure}
\newpage

\subsection{paso 16}
\label{np07:paso-16}\begin{figure}[htbp]
\centering
\capstart

\noindent\sphinxincludegraphics[width=300pt]{{16b}.jpg}
\caption{Postes Macho}\label{np07:id22}\end{figure}
\newpage

\subsection{paso 17}
\label{np07:paso-17}\begin{figure}[htbp]
\centering
\capstart

\noindent\sphinxincludegraphics[width=300pt]{{17b}.jpg}
\caption{Postes Hembra}\label{np07:id23}\end{figure}
\newpage

\subsection{paso 18}
\label{np07:paso-18}\begin{figure}[htbp]
\centering
\capstart

\noindent\sphinxincludegraphics[width=300pt]{{18b}.jpg}
\caption{Borneras}\label{np07:id24}\end{figure}
\newpage

\subsection{paso 19}
\label{np07:paso-19}\begin{figure}[htbp]
\centering
\capstart

\noindent\sphinxincludegraphics[width=300pt]{{19b}.jpg}
\caption{Conector USB hembra B}\label{np07:id25}\end{figure}
\newpage

\subsection{paso 20}
\label{np07:paso-20}\begin{figure}[htbp]
\centering
\capstart

\noindent\sphinxincludegraphics[width=300pt]{{20b}.jpg}
\caption{Capacitores Cerámicos 0,1uF}\label{np07:id26}\end{figure}
\newpage


\renewcommand{\indexname}{Index}
\printindex
\end{document}
